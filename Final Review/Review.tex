\documentclass[12pt]{article} 
\usepackage[utf8]{inputenc}
\usepackage{geometry}
\geometry{letterpaper}
\usepackage{graphicx} 
\usepackage{parskip}
\usepackage{booktabs}
\usepackage{array} 
\usepackage{paralist} 
\usepackage{verbatim}
\usepackage{subfig}
\usepackage{fancyhdr}
\usepackage{sectsty}

\pagestyle{fancy}
\renewcommand{\headrulewidth}{0pt} 
\lhead{}\chead{}\rhead{}
\lfoot{}\cfoot{\thepage}\rfoot{}

%%% SECTION TITLE APPEARANCE
\allsectionsfont{\sffamily\mdseries\upshape} 

%%% ToC (table of contents) APPEARANCE
\usepackage[nottoc,notlof,notlot]{tocbibind} 
\usepackage[titles,subfigure]{tocloft}
\renewcommand{\cftsecfont}{\rmfamily\mdseries\upshape}
\renewcommand{\cftsecpagefont}{\rmfamily\mdseries\upshape} %

\usepackage{amsmath}
\usepackage{amssymb}
\usepackage{empheq}

\renewcommand{\L}[1]{\mathcal{L}\{#1\}}

\title{Ordinary Differential Equations Cheat Sheet}
\author{Milan Capoor}
\date{Fall 2022}

\begin{document}
\maketitle

\section{The Fundamental Equation}
For equations of the form:
\[y' = k y\] 

Solution:
\[\boxed{y = Ce^{k t}}\]

\section{Classification}
\begin{enumerate}
    \item \emph{Order:} highest number of derivatives
    \item \emph{Homogeneous:} if the RHS is 0
    \item \emph{Inhomogeneous:} if the RHS is not 0
    \item \emph{Linear:} the coefficients depend on t but not y
    \item \emph{Nonlinear:} the coefficients depend on y
\end{enumerate}

\section{Solve by Separation}
For equations of the form:
\[y' = \frac{f(t)}{g(t)}\]

Solution:
\begin{enumerate}
    \item Rearrange so all y's on one side, all t's on other
    \item Cross multiply by $dx$
    \item Integrate each side (note: +C only needs to be on one side)
    \item Isolate $y$
\end{enumerate}

Example:
\[\int g(t) \; dy = \int f(t) \; dx \]

\subsection*{The Logistic Equation:}
Equations of the form:
\[y' = \alpha y (1 - \frac{y}{\beta})\]

Solution:
\[\boxed{y = \frac{\beta Ce^{At}}{1 + Ce^{\alpha t}}}\]

\section{Integrating Factors:}
For equations of the form:
\[y' + ky = f(t)\]

Solution:
\begin{enumerate}
    \item Multiply each side by $e^{kt}$
    \item Reduce the LHS to a product rule derivative
    \item Integrate both sides
    \item Isolate $y$
\end{enumerate}

\subsection*{More general integrating factors}
For equations of the form:
\[y' + P(t) y = f(t)\]

Use 
\[e^{\int P(t) \; dt}\]
as the integrating factor with the method described above

\section{Exact Equations:}
For equations of the form:
\[P(x, y)\; dx + Q(x, y)\; dy = 0\]

Solution:
\begin{enumerate}
    \item Check that $F = \langle P, Q \rangle$ is conservative ($P_y = Q_x$ -- mnemonic "Peyam is quixotic")
    \item Find the potential function f for which $F = \nabla f$ (integrate P and Q, creating a function of all terms)
    \item The solution of the corresponding ODE will simply be 
    \[\boxed{f = C}\]
\end{enumerate}

\subsection*{Non exact equations}
For nonexact equations, multiply the entire ODE by the given integrating factor (found using PDEs)

\section{Euler's Method}
The value of y at a point can be approximated:
\[y_{n + 1} \approx y_n + \left(\frac{t_n - t_0}{N}\right) \; y'(y_n, t_n)\]

\section{Modelling Higher order equations as systems}
Example: $y'' + 4y = 0$

Solution:
\[\begin{cases}
    x_1(t) = y\\
    x_2(t) = y'
\end{cases} \implies \begin{cases}
    x_1' = y' = x_2\\
    x_2' = y'' = -4y = -4x_1
\end{cases}\] 
\[\begin{cases}
    x_1' = 0x_1 + x_2\\
    x_2' = -4x_1 + 0x_2
\end{cases}\]

\[\vec{x}'(t) = \begin{bmatrix}
    0 & 1\\
    -4 & 0
\end{bmatrix} \vec{x}(t)\]

\section{Systems of ODE}
Given an equation in the form 
\[x' = Ax\]

The solution is of the form
\[\boxed{x(t) = C_1 e^{\lambda_1 \, t}\; \vec{v}_{\lambda_1} + C_2 e^{\lambda_2\, t}\; \vec{v}_{\lambda_2}}\]

\subsection*{Complex Eigenvalues}
Essential formula:
\[e^{it} = \cos t + i \sin t\]

To solve a system of ODE with complex eigenvalues $\pm  \lambda$,
\begin{enumerate}
    \item Find the eigenvector for one of the complex eigenvalues (say $\lambda$)
    \item Trick: during the gaussian elimination, choose one of the two rows to 0 and solve the other to get the eigenvector 
    \item \[e^{(\lambda i)t} \begin{bmatrix}
        x_1 + x_3 i\\
        x_2 + x_4 i
    \end{bmatrix} = (\cos (\lambda t) + i \sin (\lambda t))\left(\begin{bmatrix}
        x_1\\x_2
    \end{bmatrix} + i \begin{bmatrix}
        x_3\\x_4
    \end{bmatrix}\right)\]
    \item Distribute the above equation to get a real term $\vec{x}_1$ and an imaginary term $\vec{x}_2$
    \item The general solution will then be in the form 
    \[\boxed{\vec{x}(t) = C_1 \vec{x}_1 + C_2 \vec{x}_2}\]
\end{enumerate}

\subsection*{Repeated Eigenvalues}
To solve a system of ODE with a repeated eigenvalue $\lambda$:
\begin{enumerate}
    \item Find the first eigenvector $\vec{v}$ as normal
    \item Find the generalized eigenvector $\vec{w}$ by solving by row reduction $(A - \lambda I) \vec{w} = \vec{v}$
    \item The general solution will be in the form
    \[\boxed{x(t) = C_1e^{\lambda t} \; \vec{v} + C_2 (te^{\lambda t} \vec{v} + e^{\lambda t} \vec{w})}\]
\end{enumerate}

Note: do not rescale the generalized eigenvector

\section{Matrix Exponentials}
Note that equations in the form 
\[x' = Ax\]
are analagous to equations of the form 
\[y' = ky\]

Thus, to solve
\[\boxed{x = Ce^{At} = Pe^{Dt}P^{-1} \begin{bmatrix}
    C_1\\
    C_2
\end{bmatrix}}\]

\subsection*{Matrix Exponentials wih repeated eigenvalues}
For $2 \times 2$ matrices with one eigenvalue,
\[(A - \lambda I)^2 = 0\]

Thus, in the Taylor expansion of the matrix exponentiation, all higher terms for to zero.
\[e^{At} = e^{\lambda t}(I + (A - \lambda I)t)\]
So,
\[\boxed{x = (I + (A - \lambda I)t) \begin{bmatrix}
    C_1\\C_2
\end{bmatrix}e^{\lambda t} }\]

\section{Equilibrium points of nonlinear systems}
For systems of the form 
\[\begin{cases}
    x' = f(x, y)\\
    y' = g(x, y)
\end{cases}\]

Solution: 
\begin{enumerate}
    \item Find the nullclines ($x' = 0$ and $y' = 0$) 
    \item Solve for the equilibrium points
    \item Find the Jacobian/linearization
    \[\nabla F = \begin{bmatrix}
        f_x & f_y\\
        g_x & g_y
    \end{bmatrix}\]
    \item Classify the equilibrium points 
\end{enumerate}

\subsection*{CLassifying equilibrium points}
For the equilibrium point $(x, y)$,
\begin{itemize}
    \item If both eigenvalues of $F(x, y)$ are positive, it is unstable
    \item If both eigenvalues are negative, it is stable
    \item If one positive, one negative, it is a semistable saddle point
\end{itemize}

\section{2nd Order ODEs}
For an equation of the form, 
\[Ay'' + By' + Cy = 0\]

The auxiliary equation is
\[Ar^2 + Br + C = 0\]
with roots $r_1$ and $r_2$

The solution will then be in the form 
\[\boxed{y = C_1e^{r_1\, t} + C_2 e^{r_2\, t}}\]

\subsection*{Complex roots}
For equations with auxiliary roots of the form $a + bi$, the solution will be in the form:
\[\boxed{y = e^{at} \left(C_1 \cos(bt) + C_2 \sin(bt)\right)}\]

\subsection*{Repeated roots}
For equations with with repeated auxiliary roots, the solution will be of the form 
\[\boxed{y = C_1 e^{rt} + C_2te^{rt}}\]

\section{Boundary value problems}
Find the eigenvalues and eigenfunctions of equations of the form
\[y'' = \lambda y\]

Solution:
\begin{enumerate}
    \item Identify the auxiliary equation $r^2 = \lambda$
    \item In the case $\lambda > 1$, search for nonzero solutions where $r = \pm \omega$, for a positive number $\omega$
    \item Case $\lambda = 0$: search for nonzero solutions
    \item Case $\lambda < 0$: $r = \pm \omega i \quad (\omega > 0)$
    \item Parameterize the infinite solutions with aa positive integer $m$
\end{enumerate}
\section{Inhomogeneous equations}
Solution:
\begin{enumerate}
    \item Find the homogeneous solution $y_0$ using the auxiliary equation
    \item Find any particular solution $y_p$
    \item The general solution will then be of the form
    \[\boxed{y = y_0 + y_p} \]
\end{enumerate}

\subsection*{Undetermined coefficients}
To find the particular solution, make an initial guess based on the form of the inhomogeneous term:
\begin{itemize}
    \item if RHS is $e^{rt}$, guess $Ae^{rt}$
    \item if RHS is a polynomial, guess the general degree polynomial
    \item if RHS is sin or cos, guess $A\cos(rt) + B\sin(rt)$
\end{itemize}

The substitute the guess for y in the original ODE and solve for the constants

\subsection*{Resonance}
If the guess of the particular solution from the undetermined coefficients method has the same root(s) as the homoegenous solution, add a resonance term

Example: if the naive guess is $Ae^{rt}$ but the root $r$ coincides, guess
\[Ate^{rt}\]

\subsection*{Variation of parameters}
This is an alternative method to Undetermined coefficients where the particular solution to an equation of the form 
\[Ay'' + By' + Cy = f(t)\]
will be of the form 
\[y_p = u(t) e^{r_2\, t} + v(t) e^{r_2\,t}\]

To solve for u and v, solve the Wronskian matrix 
\[\begin{bmatrix}
    e^{r_1t} & e^{r_2 t}\\
    r_1 e^{r_1 t} & r_2 e^{r_2 t}
\end{bmatrix} \begin{bmatrix}
    u'\\
    v'
\end{bmatrix} = \begin{bmatrix}
    0\\
    f
\end{bmatrix}\]
using Cramer's rule where f is the inhomogeneous term/

Then integrate $u'$ and $v'$ to find the final coefficients.

\subsection*{Variation of parameters for matrix systems}
For equations of the form
\[x' = Ax + f\]

The method of variation of parameters will be the same but the var par equation is instead 
\[A\begin{bmatrix}
    u'\\
    v'
\end{bmatrix}= \vec{f}\]

\section{Laplace transform}
Fundamental equation:
\[\L{f(t)} = \int_0^\infty f(t) e^{-st}\; dt\]

Laplace transform table:
\begin{center} 
    \resizebox{0.7\linewidth}{!}{%
        \begin{tabular}{|c|c|}
            \hline
            $f(t)$ & $\L{f(t)}$\\
            \hline
            $1$ & $\frac{1}{s}$\\
            $e^{at}$ & $\frac{1}{s - a}$\\
            $t^n$ & $\frac{n!}{s^{n+1}}$\\
            $\sin(at)$ & $\frac{a}{s^2 + a^2}$\\
            $\cos(at)$ & $\frac{s}{s^2 + a^2}$\\
            $u_c(t)$ & $\frac{e^{-cs}}{s}$\\
            $u_c f(t-c)$ & $e^{-cs} \L{f(t)}$\\
            $e^{at} f(t)$ & $f(s - a)$\\
            $\delta(t - c)$ & $e^{-cs}$\\
            $y'$ & $s\L{y} - y(0)$\\
            $y''$ & $s^2 \L{y} - sy(0) - y'(0)$\\
            \hline
        \end{tabular}
    }    
\end{center}

\[(f\star g)(t) = \int_0^t f(t - \tau) g(\tau)\; d\tau\]

To solve an ODE with laplace transforms:
\begin{enumerate}
    \item Take the laplace of each term
    \item rearrange to the form 
    \[\L{y} = f(s)\]
    \item Manipulate the RHS into a sum of known laplace compositions
    \item Take the inverse laplace of those transforms
\end{enumerate}

\end{document}